
\documentclass{article}
\usepackage{times}
\usepackage{graphicx}

\begin{document}

% Article top matter
\title{CCxx00 Radio Stack}
\author{David Moss\\
        Rincon Research Corporation\\
		Tucson, Arizona\\
		\texttt{mossmoss at gmail dot com}}  
\date{\today} 
\maketitle

\pagebreak
\tableofcontents
\pagebreak

%%%%%%%%%%%%%%%%%%%%%%%%%%%%%%%%%%%%%%%%%%%%%%%%%%%%%%%%%%%%%%%%%%%%%%%%%%%%%%%%%%%%%

\section{Introduction}
\label{sec:introduction}

This document describes the architecture, options, and interfaces integrated
into the CC1100 / CC1101 / CC2500 (collectively CCxx00) open-source TinyOS radio stack.

The flexibility of the CCxx00 architecture, combined with well tested and characterized code, makes 
it a highly desirable choice for reliable wireless sensor network deployments.  This document
will describe how to setup the TinyOS CCxx00 radio stack to meet your system's target behavior, and
discuss the various interfaces you'll encounter in the stack and how to use them most effectively.

The CCxx00 open-source radio stack \cite{opensourceccxx00}, designed for seamless integration into the TinyOS open-source
operating system \cite{tinyos}, supports many options and add-on libraries that are useful for commercial and 
research ad-hoc wireless network deployments.  The architecture is very modular, allowing the 
developer to select options, layers, and behavior that make sense for the target deployment. 
Developers can choose options such as single radio platforms, multiple radio platforms, low 
power duty cycling communications, reliable links, acknowledgements, CSMA based channel sharing, 
duplicate packet suppression, and more. Complex forms of wireless networking can be implemented on
top of the radio stack, ranging from simple point-to-point communication to full mesh networks.


%%%%%%%%%%%%%%%%%%%%%%%%%%%%%%%%%%%%%%%%%%%%%%%%%%%%%%%%%%%%%%%%%%%%%%%%%%%%%%%%%%%%%
\section{Structure}
\label{sec:structure}
The radio stack was designed to be very modular, allowing developers to drop in or select new
layers of functionality. If multiple implementations exist for a given directory, it 
is the developer's job to modify how their platform compiles (using the .platform file
perhaps) to select the desired implementation. At a minimum, the compiler must know the location of each main directory
in the CCxx00 tree.

\subsection{Single Radio Components}
\label{sec:singleradiocomponents}
There are many components of functionality, some with more than one behavior implementation:

\begin{itemize}
	\item \emph{am} : Active Message layer. Configures the header, defines the payload, filters packets
	based on packet type.
	\item \emph{alarm} : 32 kHz alarm supporting Acks, CSMA, and fast timing.
	\item \emph{acks} : Acknowledgment layer
	  \begin{itemize}
	    \item \emph{DynamicAcks} : Dynamically turn acks on or off for each outgoing packet.
	    \item \emph{StaticAcks} : Statically keeps acks turned on for every packet, saves memory.
	  \end{itemize}
	\item \emph{crc} : Adds and verifies 32-bit software CRC's at the end of packets.
	\item \emph{csma} : Carrier Sense Multiple Access / Collision Avoidance
	  \begin{itemize}
	    \item \emph{singlesense} : Please see section \ref{sec:singlesense}.
	    \item \emph{continuoussense} : Please see section \ref{sec:continuoussense}.
	  \end{itemize}
	\item \emph{init} : Low-level radio hardware initialization code.
	\item \emph{interfaces} : All of the CCxx00-specific radio stack interfaces.
	\item \emph{link} : Packet Link Layer, helps with reliable node-to-node packet transmissions.
	\item \emph{lpl} : Low power communications implementations, discussed in greater detail in section \ref{sec:lpl}.
	  \begin{itemize}
	    \item \emph{alwayson} : The radio is always on. This uses the least amount of 
	    program memory and has the fastest throughput, but uses the most amount of energy.
	    \item \emph{bmac} : BMAC \cite{bmac} low power implementation. 
	    \item \emph{boxmac} : BoX-MAC \cite{boxmac} low power implementation, combining two other
	    layers in the radio stack:  BMAC and Packet Link. Use of this implementation
	    implies you also have the BMAC directory compiled in.
	    \item \emph{enddevice} : This is an end-device implementation that causes
	    the radio to turn on only when a packet is being sent, and turn off when
	    the packet is done sending.  This is very efficient on memory footprint, and 
	    very efficient on energy consumption. It can be combined with an external 
	    add-on layer, \emph{Pending Bit}, to allow your local node to remain active 
	    for a period of time after receiving an ack with the \emph{Pending Bit} set.
	  \end{itemize}
	\item \emph{powermanager} : Gracefully enables or disables critical communication 
	layers in the stack where the radio may be in the process of sending or receiving.
	\item \emph{radios} : Radio hardware configuration. The first CCxx00.h header file
	that is referenced will become the default radio for the platform.
	  \begin{itemize}
	    \item \emph{CC1100} : CC1100 / CC1101 configuration
	    \item \emph{CC2500} : CC2500 configuration
	  \end{itemize}
	\item \emph{receive} : Low-level packet reception module.
	\item \emph{select} : Used for forwards-compatibility with multiple radio 
	configurations.
	\item \emph{spi} : SPI bus implementation, provides access to FIFO's, registers, and 
	strobe commands.
	\item \emph{splitcontrolmanager} : Manages the SplitControl interface, preventing
	the radio from turning off when a packet is being sent anywhere in the stack.
	\item \emph{traffic} : Traffic control layer, dynamically limits the throughput of your local
	node to increase acknowledgment success rates for the entire network.
	\item \emph{transmit} : Low-level transmit module. This provides separate interfaces 
	for packet transmission and ack transmission, the difference is how aggressive it 
	tries to transmit. It also physically creates the long preambles for BMAC / BoX-MAC 
	low power communication strategies.
	\item \emph{unique} : Unique packet reception, prevents duplicate packets from 
	entering the stack. Recommend combining this with an unpublished security layer.
\end{itemize}

\subsection{Single Radio .platform Configuration}
Your platform's .platform file will tell the compiler which directories to pull in at
compile time.  The directories toward the top take precedence over the directories toward
the bottom.  If a similarly named module appears in more than one directory, the first
instance of it is used at compile time, overriding deprecated files. This may be
useful if you need to override files in the default CCxx00 stack directories with your own
code.  Alternatively, you can create a new directory with your own implementations and
comment out the old with a \#.

The \%T at the beginning of every directory automatically references the \$TOSDIR environment variable

A typical .platform file that references the ccxx00\_single radio stack for the compiler will look 
like this:

\begin{verbatim}
  %T/chips/ccxx00_single
  %T/chips/ccxx00_single/alarm
  %T/chips/ccxx00_single/am
  %T/chips/ccxx00_single/am/queue
  %T/chips/ccxx00_single/crc
  %T/chips/ccxx00_single/init
  %T/chips/ccxx00_single/interfaces
  %T/chips/ccxx00_single/link
  %T/chips/ccxx00_single/packet
  %T/chips/ccxx00_single/powermanager
  %T/chips/ccxx00_single/receive
  %T/chips/ccxx00_single/select
  %T/chips/ccxx00_single/spi
  %T/chips/ccxx00_single/splitcontrolmanager
  %T/chips/ccxx00_single/traffic
  %T/chips/ccxx00_single/transmit
  %T/chips/ccxx00_single/unique
  
  /** Choose static acks to trim memory footprint 
      and get an ack for every packet */
  %T/chips/ccxx00_single/acks/staticacks
  #%T/chips/ccxx00_single/acks/dynamicacks
  
  /** Choose CC1100 / CC1101 radio */
  %T/chips/ccxx00_single/radios/cc1100
  #%T/chips/ccxx00_single/radios/cc2500
  
  /** Choose continuous sense CSMA to support 
      better ack rates and fairness */
  %T/chips/ccxx00_single/csma/continuoussense
  #%T/chips/ccxx00_single/csma/singlesense
  
  /** Include BMAC to give the option of basic 
      low power communications */
  %T/chips/ccxx00_single/lpl
  #%T/chips/ccxx00_single/lpl/alwayson
  %T/chips/ccxx00_single/lpl/bmac
  #%T/chips/ccxx00_single/lpl/boxmac
  #%T/chips/ccxx00_single/lpl/enddevice
  
\end{verbatim}

\subsection{Multiple Radio Components}
\label{sec:multipleradiocomponents}

The CCxx00 radio stack supports multiple radios. You probably need to prevent radios 
from turning on simultaneously, because the crystals that power the radios interfere with
each other and prevent transmission / reception without very robust filters preventing
that interaction.  The CCxx00 radio stack was designed to prevent multiple radios from
turning on simultaneously.

\begin{itemize}
	\item \emph{init} : Overrides the CCxx00PlatformInit module, which puts multiple 
	radios to sleep on Platform Init.
	\item \emph{interfaces} : Provides a CentralWiring interface to support multiple 
	radios.
	\item \emph{select} : Implements the Radio Select layer, which allows the user to 
	select which radio a packet should be sent through, and which radio is enabled.
	\item \emph{spi} : BlazeCentralWiring to support switching between one 
	radio's physical interfaces to the next.
\end{itemize}

\subsection{Multiple Radio .platform Configuration}
To enable multiple radios, modify your .platform file to include the ccxx00\_multiple directories at the top, which will override files from the ccxx00\_single directories.

\begin{verbatim}
  %T/chips/ccxx00_multiple/
  %T/chips/ccxx00_multiple/init
  %T/chips/ccxx00_multiple/interfaces
  %T/chips/ccxx00_multiple/select
  %T/chips/ccxx00_multiple/spi
\end{verbatim}

\subsection{CCxx00 Add-ons}
\label{sec:ccxx00addons}
The radio stack was designed for flexibility and modularity. Some might even claim it is too flexible, but all of that flexibility goes a long way toward allowing developers to add or modify functionality in the radio stack to meet their system needs. 

Of course, there is a balance to strike between flexibility and memory footprint, and our goal was to make the tightest radio stack that is 
reliable and provides the basic foundation for mesh networking functionality. By placing add-on functionality in a separate space than the basic radio stack, we can keep the program footprint to a minimum while giving the option of including
more functionality.

At the time of this writing, the CCxx00 radio stack had the following add-ons available to developers:

\begin{itemize}
	\item \emph{availableradios} : Discovers which radios are connected to 
	a platform during initialization. This taps directly into the 
	CCxx00PlatformInit module to send a few more commands to the radios on 
	board. The purpose was to verify platform functionality to prevent 
	libraries from accessing non-existent radios on a modular radio
	platform.
	
	\item \emph{dutycycle} : Keeps track of the amount of time the radio has been in some active state vs. how long the platform has been enabled. This is extremely useful for energy consumption estimates.
	\item \emph{packetlink} : This is the Packet Link implementation,
	replacing the stubbed out version in the ccxx00\_single main radio 
	stack.
	\item \emph{packetstats} : Keeps track of the number of packets sent, received, and overheard. As of the time of this writing, we have yet to implement another statistics interface to provide the number of missed acknowledgments.
	\item \emph{pendingbit} : This layer watches out for a pending bit enabled in the FCF word of the packet and will activate the radio for a period of time if a pending bit is ever seen. This could be used in conjunction with any low power communication profile except always-on, where it wouldn't make sense because the radio is always active.	
	\item \emph{rssi} : Ambient RSSI gathering component, where you can
	read the RSSI value without having received a packet.
	\item \emph{trafficcontrol} : Traffic Control implementation, replacing
	the stubbed out version in the ccxx00\_single main radio stack.
\end{itemize}


\subsection{Radio Add-ons .platform Configuration}
To enable add-ons radios, two things must occur. First, you must modify your .platform file to include the ccxx00\_addons directories. Second, you must reference the add-on configuration file(s) to pull it in at compile time. The normal coding
strategy is to modify your platform's ActiveMessageC.nc configuration to pull in the add-ons and provide their interfaces
out of the top of ActiveMessageC.

\begin{verbatim}
  %T/chips/ccxx00_addons/availableradios
  %T/chips/ccxx00_addons/dutycycle
  %T/chips/ccxx00_addons/packetlink
  %T/chips/ccxx00_addons/packetstats
  %T/chips/ccxx00_addons/pendingbit
  %T/chips/ccxx00_addons/rssi
  %T/chips/ccxx00_addons/trafficcontrol
\end{verbatim}

\subsection{Layer Wiring}
Layers stack up to form the radio driver, hence the name \'Radio Stack\'. Commands
such as SplitControl, Send, and Receive propagate through each layer serially.

All wiring is done at the highest level BlazeC.nc configuration file. You can 
change the architecture of the radio stack by adding \'dummy\' pass-through configurations
for any given layer, or hacking the BlazeC.nc file (which should be avoided to maintain
backwards / forwards compatibility if possible).

The order of these layers is very important to forming a fully functional radio stack.

\begin{table}
	\centering
		\begin{tabular}{|c|}
		  \hline
      ActiveMessageC \\ \hline \hline
      SplitControlManager \\ \hline
      Traffic Control Add-on \\ \hline
      RadioSelect \\ \hline
      UniqueReceive \\ \hline
      Packet Link Add-on \\ \hline
      Low Power Communications \\ \hline
      Acknowledgments \\ \hline
      CSMA \\ \hline \hline
      Transmit / Receive / Init \\ \hline
      SPI Bus / Central Wiring \\ \hline
      \hline
      Hardware Presentation Layer \\ \hline
      Platform Code \\ \hline
      \hline
      Hardware \\
      \hline
      
   	\end{tabular}
	\caption{CCxx00 Radio Stack}
	\label{tab:CCxx00RadioStack}
\end{table}


\begin{verbatim}  
  /***************** Send Layers ****************/
  AMSend = BlazeActiveMessageC;
  BlazeActiveMessageC.SubSend -> TrafficControlC.Send;
  TrafficControlC.SubSend -> RadioSelectC.Send;
  RadioSelectC.SubSend -> SplitControlManagerC.Send;
  SplitControlManagerC.SubSend -> PacketLinkC.Send;
  PacketLinkC.SubSend -> LplC.Send;
  LplC.SubSend -> AcknowledgementsC.Send;
  AcknowledgementsC.SubSend -> CsmaC;
  
  /***************** Receive Layers ****************/
  Receive = BlazeActiveMessageC.Receive;
  Snoop = BlazeActiveMessageC.Snoop;
  BlazeActiveMessageC.SubReceive -> RadioSelectC.Receive;
  RadioSelectC.SubReceive -> UniqueReceiveC.Receive;
  UniqueReceiveC.SubReceive -> LplC.Receive;
  LplC.SubReceive -> BlazeReceiveC.Receive;
    
  /***************** SplitControl Layers ****************/
  SplitControl = RadioSelectC.SplitControl;
  RadioSelectC.SubControl -> SplitControlManagerC.SplitControl;
  SplitControlManagerC.SubControl -> LplC.SplitControl;
  LplC.SubControl -> Ccxx00PowerManagerC.SplitControl;
\end{verbatim}



%%%%%%%%%%%%%%%%%%%%%%%%%%%%%%%%%%%%%%%%%%%%%%%%%%%%%%%%%%%%%%%%%%%%%%%%%%%%%%%%%%%%%
\section{Interfaces}
\label{sec:interfaces}
There are many available interfaces to access in the CCxx00 stack, and most are 
located in the ccxx00\_radio/interfaces directory.  We will cover
these interfaces, starting from the core interfaces to extended and advanced interfaces.

%%%%%
\subsection{Basic Core Interfaces}
\label{sec:coreinterfaces}
These are the basic interfaces you need to turn on the radio, send, and receive packets.

\subsubsection{SplitControl}
The SplitControl interface is a split-phase call to activate or deactivate a radio.
You must activate the radio before using it, and the radio is only considered ready for
use when the SplitControl.startDone() event is signaled.

When active, the radio is expected to transmit or receive packets at any time. 
When your platform boots, a call to SplitControl.start() should never return FAIL.  
This can prevent you, the developer, from wasting precious code space trying to test 
for SUCCESS on SplitControl.start() at boot.

The desired behavior of SplitControl is documented in TEP 115 \cite{tep115}. Some minor alterations
to the implementation of this TEP may have been made in the SplitControlManager component
of the radio stack to decrease compiled code size, since the user is generally only interested
in SUCCESS or FAIL of enabling / disabling the radio stack. 

When low power duty cycling is enabled with BMAC, BoX-MAC, or End Device, the physical 
radio may be off most of the time, but the radio can still send and receive packets.

\subsubsection{AMSend}
This is the basic Active Message Send interface, defined by the TinyOS community.

\subsubsection{Receive}
This is the basic Receive interface, defined by the TinyOS community. One modification
from Receive's original design is the fact that the CCxx00 stack will not double-buffer packets
at the lowest layer. The reason it was implemented this way is because the low level
double buffering functionality has never been proven useful, and the inclusion of 
it wastes program ROM.

If you need to do double-buffering for any reason, memcpy the inbound packet's contents to
your own array, post a task to do processing on it at a later time, and then return a message 
pointer as usual. Returning any pointer, including NULL, will not alter any behavior in the 
functioning radio stack.

%%%%%
\subsection{Packet Interfaces}
These interfaces tap directly into a given packet to retrieve information about the packet's
header, payload, or footer. 

\subsubsection{AMPacket}
This interface, defined by the TinyOS community, allows access to standard fields in the
header of a packet. These fields include the source address, destination address, packet
type, and group ID. 

If you set any of these fields and send the packet, the fields will be automatically
overwritten by the radio stack. In that sense, the AMPacket interface is more useful for
obtaining these fields in a given packet.  The AMPacket interface also returns the local node's
software address with the address() command. 

To set the node's address, use the ActiveMessageAddressC component, located in the tos/system directory in the 
TinyOS baseline. To make your node remember its software-configured address after a reboot 
event, you must store and load it from non-volatile memory. One easy option is to learn how to use the 
Configurator component provided by Rincon in the tinyos-2.x-contrib/rincon tree. This 
component mirrors small variables normally managed in RAM on your microcontroller's 
internal non-volatile memory. 

\subsubsection{Packet}
The Packet interface, also defined by the TinyOS community,
is mostly concerned with the payload of an inbound or outbound packet.

\subsubsection{BlazePacket}
This interface provides access to CCxx00-specific fields that may not be found in other
radio stacks.  This includes commands to look at LQI, RSSI, and the Packet Pending bit.

%%%%%
\subsection{Extended Interfaces}
\label{sec:extendedinterfaces}

\subsubsection{TrafficControl / TrafficPriority}
Networks with high traffic rates perform poorly in terms of acknowledgment success rate
than a network that has built in traffic control mechanisms. Although
some mesh networking libraries, such as CTP, have a traffic control implementation
built in, we've taken it a step further and applied it to the radio stack as a whole.
This way, one rogue element of your system won't clobber the other nodes in the area.
Network acknowledgmenet success rate goes up, and general energy consumption goes down.
The Traffic Control delays can also be configured to your radio's data rate and CSMA settings.

Traffic Control will dynamically adjust the time between packet transmissions from your local
node. By default, it initializes to a long delay between packets (4 seconds).  If acknowledgments are succeeding,
the delay decreases.  If acknowledgments fail, the delay increases.  If the delay between
packets reaches the minimum or maximum amounts, the delay will reset to 4 seconds. This is
similar behavior to TCP traffic control.

If you need a packet to get through quickly, the TrafficPriority interface can configure an outbound packet 
to be sent very quickly, bypassing the traffic control delays.

The Traffic Control layer is available as an add-on implementation that must be pulled in at compile time to override
the ccxx00\_single\\traffic dummy implementation.

\subsubsection{PacketLink}
Packet Link allows you to configure a packet to be retried for a set number of times before
the radio stack gives up sending it. Acknowledgments are automatically set when Packet Link
is enabled. In the case where a packet is received by the destination node, but the acknowledgment
is dropped, the follow on duplicate packets will automatically be filtered out by the receiving 
radio stack's UniqueReceive filter.

The Packet Link layer is available as an add-on implementation that must be pulled in at compile time to
override the ccxx00\_single\\packetlink dummy implementation.

See TEP 127 \cite{tep127} for more details.

\subsubsection{PacketAcknowledgements}
Yes, this interface is misspelled in an American English sense, due to its history. 

If Static Acknowledgments is used, this interface only serves to check the status of
an acknowledgment.  

If the Dynamic Acknowledgments implementation is used, you must explicitly call this 
interface to request an acknowledgment before the receiving node will send one back.

\subsubsection{SendNotifier}
The SendNotifier interface always generates a signal when the radio stack is about to send
a packet. The interface is parameterized by AM ID out the top of the CCxx00 stack. 

This is useful because it allows other areas of your system to modify outbound packets. 
For example, the Collection Tree Protocol connects directly into the radio stack and will not allow you
to configure outbound packets to be delivered to low power receivers. Instead of modifying
the CTP code to do the configuration there, you can add a module into the system that will
look for outbound CTP packets and configure them to be delivered to a low power receiver.

If you do not implement the SendNotifier interface anywhere, it should be removed by the
compiler which costs no code space.

\subsubsection{AckSendNotifier}
Same as the SendNotifier interface, but only pertains to acknowledgment frames that are 
about to be returned. This could, for example, be used to set the Packet Pending bit
in an outbound acknowledgment. In the future, we may modify the Receive component to
also give the option to prevent sending back an acknowledgment to unauthorized nodes.

\subsubsection{LowPowerListening}
The CCxx00 radio stack implements the latest LowPowerListening interface. This interface
will configure the wake-up intervals for your local node, and configure individual outbound packets
to be delivered to a duty cycling receiver. Please see TEP 105 \cite{tep105} for details on the original implementation. 

\subsubsection{SystemLowPowerListening}
This new interface requires support from components external to the radio stack. You can 
set the desired wake-up interval for all outbound packets. The external module that does
not exist as part of the CCxx00 stack is supposed to provide an AMSend interface, which
when you send a packet through it, will configure that packet with default low power 
communications settings. 

%%%%%
\subsection{Advanced Interfaces}
\label{sec:advancedinterfaces}

The following extremely advanced interfaces are not normally used, but do allow for 
very detailed control over the operation of the radio stack.

\subsubsection{Csma}
Our CCxx00 radio stack implements CSMA/CA to share the channel fairly with other nearby
nodes. When a packet enters a CSMA layer, this interface will signal a request to see
if any part of the application even wants to use clear channel assessments to avoid packet collisions.
In some point-to-point isolated cases, it may make sense to disable clear channel assessments.
This will allow your node's transmissions to occur as fast as possible.

The CSMA layer, as well as the effects of its backoffs, are discussed in section \ref{sec:csma}.

\subsubsection{Backoff as InitialBackoff}
This interface, parameterized by AM ID, will allow your application to configure initial
CSMA backoff settings for the current outbound packet. Initial backoffs occur for every outbound
packet when CSMA is enabled. Modifying the duration of an initial backoff could be useful, for example,
if the outbound packet is part of a large block of data that needs to be burst through
to another point very quickly.  

Modifying the backoff settings effectively changes the priority of a packet on the channel.
Shorter backoffs equal higher priority.

\subsubsection{Backoff as CongestionBackoff}
Congestion backoffs occur if your node attempts to transmit, but finds the channel busy.
You can modify the packet's priority on the channel by increasing or decreasing its backoff
from the default values the radio stack selects.

%%%%%
\subsection{Multiple Radio Interfaces}
\label{sec:multipleradiointerfaces}
These interfaces are only used when the ccxx00\_multiple behavior is compiled in.

\subsubsection{SplitControl as BlazeSplitControl[radio\_id\_t radioId]}
This SplitControl interface will allow the user to attempt to turn a different radio
on or off. Remember, only one radio is allowed to be turned on at a time.

\subsubsection{RadioSelect}
This interface will allow the user to select which radio a packet is supposed to be
sent through, and also informs the application to know which radio the packet came from.


%%%%%%%%%%%%%%%%%%%%%%%%%%%%%%%%%%%%%%%%%%%%%%%%%%%%%%%%%%%%%%%%%%%%%%%%%%%%%%%%%%%%%
\section{Hardware Presentation Layer}
\label{sec:HardwarePresentationLayer}

The CCxx00 radio stack can port to any platform, with general IO connections to any
general IO port.  The hardware presentation layer (HPL), which is platform-specific and outside
of the radio stack, provides interfaces that allows the stack to interoperate with hardware.

There are several main files the CCxx00 radio stack expects to find. Typically, the developer
will place these files in a directory such as /tos/platforms/yourplatform/chips/ccxx00
to declare they are platform-specific configurations related to the CCxx00 chip.

It is your responsibility to implement these configurations and interfaces for your platform,
and modify your .platform file so the compiler knows where to find the HPL directory.
The radio stack itself will reference these configuration files, which will cause them
to be pulled in at compile time.

\subsection{HplCC1100PinsC / HplCC2500PinsC}
This file is pulled in by the CC1100ControlC or CC2500ControlC.  It provides 
GeneralIO and GpioInterrupt interfaces which allow the radio stack to interact
with the radio's CSN and GDOx digital IO lines.

The signature for this hardware presentation layer configuration looks like this:

\begin{verbatim}
configuration HplCC1100PinsC {

  provides interface GeneralIO as Csn;
  provides interface GeneralIO as Gdo0_io;
  provides interface GeneralIO as Gdo2_io;
  
  provides interface GpioInterrupt as Gdo2_int;
  provides interface GpioInterrupt as Gdo0_int;
  
}
\end{verbatim}


\subsection{HplRadioAlarmC}
This is a generic configuration that creates a new instance of the Alarm component.
Each new instance should create a unique client for the alarm, where one client's 
start() command will not eventually result in a different client's fire() event.


\begin{verbatim}
generic configuration HplRadioAlarmC() {

  provides interface Init;
  provides interface Alarm<T32khz,uint16_t> as Alarm32khz16;

}
\end{verbatim}

An example implementation for an MSP430 platform could look like this:

\begin{verbatim}

implementation {

  components new Alarm32khz16C();

  Init = Alarm32khz16C;
  Alarm32khz16 = Alarm32khz16C;
  
}
\end{verbatim}



\subsection{HplRadioSpiC}
The HplRadioSpiC configuration simply forwards TinyOS SPI-bus interfaces to the 
appropriate SPI bus module.

\begin{verbatim}
configuration HplRadioSpiC {
  
  provides interface Resource;
  provides interface SpiByte;
  provides interface SpiPacket;
  
}
\end{verbatim}

An example implementation for an MSP430 platform could look like this:

\begin{verbatim}
implementation {

  components new Msp430Spi0C() as SpiC;
  Resource = SpiC;
  SpiByte = SpiC;
  SpiPacket = SpiC;

}
\end{verbatim}


\subsection{platform\_message.h}

You need to modify your platform\_message.h file to include the CCxx00 radio stack headers,
footers, etc. in the definition of message\_t.

\begin{verbatim}
#ifndef PLATFORM_MESSAGE_H
#define PLATFORM_MESSAGE_H

#include <Blaze.h>
#include <Serial.h>

typedef union message_header {
  blaze_header_t blaze;
  serial_header_t serial;
} message_header_t;

typedef union TOSRadioFooter {
  blaze_footer_t blaze;
} message_footer_t;

typedef union TOSRadioMetadata {
  blaze_metadata_t blaze;
} message_metadata_t;

#endif
\end{verbatim}


%%%%%%%%%%%%%%%%%%%%%%%%%%%%%%%%%%%%%%%%%%%%%%%%%%%%%%%%%%%%%%%%%%%%%%%%%%%%%%%%%%%%%
\section{Layers and Architecture Details}
\label{sec:layersandarchitecturedetails}
In this section, we'll comment on some of the layers and behavior in the radio stack,
and clarify which alternative implementations for a single layer may be better for your
system than others.

\subsection{SPI Bus Implementation}
\label{sec:spibus}
The CCxx00 SPI bus was built from the beginning to support access to every register,
strobe, and FIFO, which would allow the developers to experiment with all available 
options.

There are two layers of SPI bus arbiters built into the system. First, there is
the SPI bus arbiter for the entire platform, which is built into TinyOS. On top of 
this arbiter is a CCxx00 driver specific arbiter which allows individual layers 
in the radio stack to control the SPI bus exclusive of each other.
In order for an individual layer to access the SPI bus, it only has to call one 
request() command, which is provided by instances of BlazeSpiResourceC. 

Multiple layers can request access to the SPI bus at a single time. The CCxx00
SPI bus arbiter will grant access in a round robin fashion.  When there are no
more users, the CCxx00 SPI bus arbiter will signal a request to see if it
should temporarily maintain control of the platform's SPI bus. This is useful,
for example, when the CCxx00 driver is waiting for an acknowledgment but no layers
necessarily need to be interacting with the radio hardware.

When access is SPI granted to a layer in the radio stack, it is up to that layer
to clear / set the chip select (CSN) line before attempting to interact with the
radio.

\subsection{Blaze Central Wiring}
\label{sec:centralwiring}
The CCxx00 radio stack supports individual as well as multiple radios.  The 
central wiring component simply brings all of the hardware and configuration
interfaces for each individual radio into the same location.  In this manner,
the BlazeCentralWiringC component simply acts as a fa\c{c}ade to connect with
more specifically named components like CC1100ControlC, CC2500ControlC, etc.


\subsection{Acks}
\label{sec:acks}
\subsubsection{General notes on acknowledgments}
The ack layer actually sits on top of CSMA and matches the source address,
destination address, and data sequence number (DSN) of any incoming acknowledgment
with corresponding fields in the outbound packet.

When a packet enters the Acknowledgment layer, the packet is quickly forwarded
onto the CSMA layer. The acknowledgment timeout window does not start until the 
packet has actually been transferred, which is indicated by the sendDone() event
coming back from CSMA.  At that time, the ack layer starts its wait timer, which 
is really a 32 kHz alarm for more accuracy, to wait for an ack to be received by the radio.

The CSMA layer will attempt to release the SPI bus resource, and the CCxx00's personal SPI
bus arbiter will ask any listeners if it should release the SPI bus for the entire radio.
The ack layer will refuse to let go of the SPI bus for the chip.  This allows ack timing constraints
to be met. Imagine what would happen if we started waiting for an ack, but the flash chip driver
took over the SPI bus and didn't release it until after our ack window: the ack would be dropped.
This strategy prevents that scenario from occurring. 

The acknowledgment wait period is in units of jiffies (1 jiffy is 1/32768 clock ticks
per second, approximately 30.51 microseconds) and depends on your data rate.  
Its default value is a \#define in the Acknowledgements.h
header file, which can be overridden elsewhere to be compatible with your chosen data rate. 
For very fast data rates, the ack wait period could be on the order of 200-300 jiffies.  At slow
data rates, it could be as long as 2000 jiffies or more.  The longer the ack wait period, the more
likely you will get a successful ack back.  The shorter the ack wait period, the faster your
throughput.

To discover a good ack wait period for your given data rate, you calculate it, or setup a test to transmit
packets from a receiver to a transmitter and increase the ack wait period until acknowledgments
are being reliably received.  Another method, which I prefer, is modify the Blaze.h
header file to set ENABLE\_BLAZE\_TIMING\_LEDS to 1 and monitor the LED lines on a logic analyzer
across several platforms that are all transmitting and requesting acknowledgments. This lets the
developer actually see on the logic analyzer when a platform is transmitting, when it is waiting
for an ack, and when the other platform is actually sending the ack. We should see the ack wait
period be slightly longer than the amount of time it takes the receiving platform to transmit
back an ack.

There are two things that might cause an ack to drop. First, collisions. Because there is a slight
gap in modulation between the time the transmission ends and the time the acknowledgment begins to 
transmit, another node may jump in and start transmitting on the channel before or during the ack.
This could push the ack outside the bounds of the ack window, or straight up collide. This could 
potentially be improved by leaving the original transmitter in TX mode (instead of going back
into RX mode) and leaving TX mode manually. The purpose of this would be to remove the gap in
modulation between the packet transmission and the ack transmission.

The second issue that has been observed is the fact that the CCxx00 radio has a problem with managing
its internal state, or may see false energy on the channel when there is none. A receiver tells the
radio to go into TX mode to send the ack, but the physical radio refuses. Eventually, the driver 
gives up and reboots the radio, which has been found to be the only solution. The reboot is 
transparent to other areas of the system, but causes the ack to be dropped because it misses its window.

\subsubsection{Dynamic Acks}
\begin{itemize}
  \item Advantages: Very flexible. Fully implements the PacketAcknowledgements interface. 
  Allows acknowledgments to be requested or not requested on a per-packet basis.
  \item Disadvantages: Uses more ROM to support flexibility.
\end{itemize}

\subsubsection{Static Acks}
\begin{itemize}
  \item Advantages: Smaller memory footprint. All packets request acknowledgments.
  \item Disadvantages: Cannot prevent a packet from being acknowledged, which could 
  decrease throughput if you don't need to acknowledge on a per-packet basis.
\end{itemize}

\subsection{CSMA}
\label{sec:csma}
CSMA/CA allows your node to share the channel fairly with other nodes.
This is the main throughput bottleneck in the system, which impacts
throughput even more so than your data rate in most cases. 

The Csma.h files define the amount of time, in jiffies, the backoff
events should occur.  Backoff periods are also randomized to some
degree to prevent two nodes from attempting to transmit simultaneously.
If you need to increase your node-to-node throughput, decrease the minimum backoff periods. Beware though: decreasing the backoff periods
too much in a congested setting will cause collisions that will actually
decrease total throughput. 

\subsubsection{Single Sense}
\label{sec:singlesense}
Single Sense CSMA mimics the 802.15.4 CSMA implementations, like those
found in the CC2420 stack.  Its strategy is to backoff for a random
period of time, and then perform a clear channel assessment one time immediately before taking over the channel and transmitting.

\begin{itemize}
  \item Advantages: Low memory footprint. Faster throughput.
  \item Disadvantages: Can cause serious collisions and problems in a 
  congested network. The implementation is considered \'rude\'. 
  Completely incompatible with packetized low power communication wake-up 
  transmissions, as we've seen in BoX-MAC low power communication 
  strategies and CC2420 implementations in the past. Nodes can 
  jump in very easily between the end of a packet transmission
  and the beginning of an ack transmission, killing acks. Only use this
  CSMA implementation if you expect isolated nodes in a non-mesh
  setting.
\end{itemize}

\subsubsection{Continuous Sense}
\label{sec:continuoussense}
Continuous sense was designed and implemented in the CCxx00 stack
specifically to fix the problems with the original single sense CSMA implementation that was instigated by 802.15.4.

Continuous sense truly behaves more like humans communicate: Listen
for a period of time, if someone is heard talking during a listen period,
back off for longer but grow a little more impatient. Don't jump
in talking immediately as another entity finishes a sentence. Never interfere with acknowledgments.

The backoffs start with long durations, which gives the node a lower
priority on the channel. If energy is heard on the channel at any point
during a backoff, an energy interrupt fires and a new backoff period 
must be performed. A packet will only begin transmitting when an 
entire backoff period expires without energy on the channel at any point
during that time. The instantaneous clear channel assessments of the
single sense implementation are all done away with, and there is no way
to interrupt an acknowledgment unless the minimum backoff settings are too low or the previous transmission on the channel was out of range.

\begin{itemize}
  \item Advantages: Very robust, helps nodes share the channel
  more effectively. Much fewer collisions, much greater ack success
  rate in a network. Much more compatible with packetized wake-up
  transmissions in BoX-MAC type low power communication strategies.
  \item Disadvantages: Slightly larger memory footprint, which I find
  to be an acceptable trade-off for the increase in reliability.
\end{itemize}

\subsection{Low Power Communications}
\label{sec:lpl}
The always-on, BMAC, BoX-MAC, and End Device implementations were briefly 
described in section \ref{sec:singleradiocomponents}.  There are a few more things
to note about low power communications.  

\subsubsection{BMAC}
\label{sec:bmac}
BMAC \cite{bmac} is configured by
your application layer to wake up periodically and sample the channel
for energy.  It is a bit more sophisticated than blind energy sampling, 
and can actually prevent itself from locking up in an infinite loop
while the channel is being jammed by another device.

When BMAC wakes up to sample the channel for nearby wake-up transmissions,
it goes through two phases to save energy and prevent jamming. First,
it resets a counter and increments the count each time a sample confirms a
wake-up transmission on the channel. Every sample that observes no wake-up 
transmissions on the channel decrements the counter, and if the counter reaches
0, the radio turns off.  The initial value of this counter is selected to 
turn off the radio if the first sample returns false.

If energy is being continuously heard on the channel, this counter will increment
to the point where it exceeds a predefined threshold. Going over this threshold
causes BMAC to change sampling strategies.  The next phase of sampling looks for
for a preamble quality threshold (PQT) instead of clear channel assessments.
This action prevents most types of jamming. If the signal on a channel is not
that of a wake-up preamble, the counter will decrement to 0 and the radio will
shut off.

\subsubsection{BoX-MAC}
\label{sec:boxmac}
The BoX-MAC \cite{boxmac} implementation is a combination of the best attributes of two other
low power communication strategies:  BMAC and XMAC.  BMAC uses a long wake-up 
preamble during its transmission, which allows the channel to be continuously 
modulated and the receive checks to be extremely efficient.  XMAC uses a packet-based
wake-up transmission instead of a long preamble, which allows acknowledgments within
the wake-up transmission. The wake-up transmission can be cut short, which increases 
throughput and saves energy at the transmitter. 

Normally, the acknowledgment gaps within a packetized wake-up transmission cause 
quick and efficient receive checks in BMAC to miss the energy on the channel. The
BoX-MAC implementation in the CCxx00 stack handles this a bit differently: Each
wake-up packet actually has a long preamble, which allows the receive checks on the
receiver side to remain extremely efficient because the probability of waking up 
in an acknowledgment gap is low.

BoX-MAC simply ties together the BMAC implementation with the Packet Link layer
by using the SendNotifier interface to modify the packet as it enters the radio
stack. It is up to the developer to wire the SendNotifier interface from the radio stack
to the BoX-MAC layer, because the developer may want to make initialize outbound packets using the
SendNotifier interface before the BoX-MAC layer takes over:

\begin{verbatim}
  BoxmacC.SendNotifier -> BlazeC.SendNotifier;
\end{verbatim}

The BoX-MAC layer makes several modifications to a packet:

\begin{enumerate}
  \item Looks at the low power communication settings of a packet. If the destination's
  wake-up interval is above a certain value, BoX-MAC will divide the wake-up interval
  by a factor of 4.
  \item Multiplies the Packet Link retries by 4 to make up for the shortened preamble
  length in each individual outbound copy of the packet.
  \item Disables clear channel assessments in the CSMA layers for all copies of the
  packet except the first copy.
\end{enumerate} 

\subsubsection{Wake-on Radio}
\label{sec:wor}
Wake-on Radio \cite{cc1101datasheet} was attempted early on in the development of the CCxx00 stack. After
fighting it for some time, we had a basic implementation working on a desk. Taking it
outside proved we couldn't communicate more than the range of a desk. This was probably
the way the implementation was exercising the radio, and not the radio itself.

Some severe non-range related issues were also observed where a radio would clearly 
enter WoR mode and never come out of it. Obviously if the radio doesn't do its job, 
the microcontroller has no way of knowing anything is wrong.

We abandoned WoR after discussing it with TI. Software controlled radio duty cycling is 
more accurate and easier to control. Eventually, the last remnants of the original WoR 
implementation were removed from the stack. 

Reliability always wins over a slight increase in energy efficiency.

\subsection{Radios}
\label{sec:radios}
The radios can be configured to enable or disable address filtering, personal area network
ID filtering, and auto acknowledgments at compile time using the preprocessor flags:

\begin{verbatim}
  NO_ACKNOWLEDGEMENTS
  
  NO_ADDRESS_RECOGNITION
  
  NO_PAN_RECOGNITION
\end{verbatim}

These settings can also be modified through the BlazeConfig interface.

All registers are stored in RAM so they may be updated at any time before burst initialization
of the radio. 

\subsection{Unique Packet Filter}
\label{sec:unique}
The unique packet filter keeps track of a short history of recently seen data sequence numbers (DSN) and source
addresses. If a received packet's source address and DSN matches historical data, it is deemed to be a duplicate
and is dropped.  

Historically, the DSN field is set at the top level of the radio stack and is always incremented
linearly for each outbound packet.  Lower layers in the radio stack, such as packet link or low power communications,
may try sending multiple copies of a packet which can be identified as duplicates by a receiver because the 
DSN doesn't change.

This linearly increasing DSN implementation leaves nodes open to attack, when no other security means
are employed. Someone only needs to walk into the network, sniff a packet, and rebroadcast the packet
with a DSN incremented by one. Receiving nodes then filter out packets from the good node, while
only accepting the attacker's packets.

To prevent this kind of simple attack, the CCxx00 stack uses a random DSN for each outbound packet.
This is more unpredictable and can help prevent an attacker from controlling communications. 


%%%%%%%%%%%%%%%%%%%%%%%%%%%%%%%%%%%%%%%%%%%%%%%%%%%%%%%%%%%%%%%%%%%%%%%%%%%%%%
\section{CCxx00 Register Settings}
\label{sec:Ccxx00RegisterSettings}
Chip register values can be modified by altering the CC1100.h or CC2500.h files. We used
SmartRF studio combined with manual hacking to come up with the register values found
in these files.

You may want to modify the register settings to change the default channel or frequency,
alter the data rate, modify the modulation, etc.  Here is the process:

\begin{enumerate}
	\item Open up SmartRF studio.  Select the CC1101, for example.
	\item Configure the main settings on the first screen like you want them: frequency,
	data rate, output power, modulation, manchester, crystal frequency, etc.
	\item Go to File, Export CC1101 registers, and save the settings in a .txt file.
	\item Manually translate the new register settings into the CC1100.h file.
\end{enumerate}

There are some register settings you probably shouldn't alter, because parts of the 
radio stack were designed expecting some functionality to be setup on the radio.
Typically, registers that have been manually configured have a comment associated with them.
To help you know which registers are ok to touch and which may not be ok to touch, 
we've created a list below.

\begin{center}
\textbf{OK TO TOUCH}
\begin{itemize}
	\item \texttt{CC1100\_DEFAULT\_CHANNEL} (translates into a register)
	\item \texttt{CC1100\_DEFAULT\_FREQ2} (translates into a register)
	\item \texttt{CC1100\_DEFAULT\_FREQ1} (translates into a register)
	\item \texttt{CC1100\_DEFAULT\_FREQ0} (translates into a register)
	\item \texttt{CC1100\_PA} (translates into a register)
	\item \texttt{CC1100\_CONFIG\_SYNC1}
	\item \texttt{CC1100\_CONFIG\_SYNC0}
	\item \texttt{CC1100\_CONFIG\_FSCTRL1}
	\item \texttt{CC1100\_CONFIG\_FSCTRL0}
	\item \texttt{CC1100\_CONFIG\_MDMCFG4}
	\item \texttt{CC1100\_CONFIG\_MDMCFG3}
	\item \texttt{CC1100\_CONFIG\_MDMCFG2}
	\item \texttt{CC1100\_CONFIG\_MDMCFG1}
	\item \texttt{CC1100\_CONFIG\_MDMCFG0}
	\item \texttt{CC1100\_CONFIG\_MCSM2}
	\item \texttt{CC1100\_CONFIG\_FOCCFG}
	\item \texttt{CC1100\_CONFIG\_BSCFG}
	\item \texttt{CC1100\_CONFIG\_AGCTRL2}
	\item \texttt{CC1100\_CONFIG\_AGCTRL1}
	\item \texttt{CC1100\_CONFIG\_AGCTRL0}
	\item \texttt{CC1100\_CONFIG\_WOREVT1}
	\item \texttt{CC1100\_CONFIG\_WOREVT0}
	\item \texttt{CC1100\_CONFIG\_WORCTRL}
	\item \texttt{CC1100\_CONFIG\_FREND1}
	\item \texttt{CC1100\_CONFIG\_FREND0}
	\item \texttt{CC1100\_CONFIG\_FSCAL3}
	\item \texttt{CC1100\_CONFIG\_FSCAL2}
	\item \texttt{CC1100\_CONFIG\_FSCAL1}
	\item \texttt{CC1100\_CONFIG\_FSCAL0}
	\item \texttt{CC1100\_CONFIG\_RCCTRL1}
	\item \texttt{CC1100\_CONFIG\_RCCTRL0}
	\item \texttt{CC1100\_CONFIG\_FSTEST}
	\item \texttt{CC1100\_CONFIG\_PTEST}
	\item \texttt{CC1100\_CONFIG\_AGCTST}
	\item \texttt{CC1100\_CONFIG\_TEST2}
	\item \texttt{CC1100\_CONFIG\_TEST1}
	\item \texttt{CC1100\_CONFIG\_TEST0}
\end{itemize}
\end{center}




\begin{center}
\textbf{DO NOT TOUCH}
\begin{itemize}
	\item \texttt{CC1100\_CONFIG\_IOCFG2}
	\item \texttt{CC1100\_CONFIG\_IOCFG1}
	\item \texttt{CC1100\_CONFIG\_IOCFG0}
	\item \texttt{CC1100\_CONFIG\_FIFOTHR}
	\item \texttt{CC1100\_CONFIG\_PKTLEN}
	\item \texttt{CC1100\_CONFIG\_PKTCTRL1}
	\item \texttt{CC1100\_CONFIG\_PKTCTRL0}
	\item \texttt{CC1100\_CONFIG\_ADDR}	
	\item \texttt{CC1100\_CONFIG\_CHANNR}
	\item \texttt{CC1100\_CONFIG\_FREQ2}
	\item \texttt{CC1100\_CONFIG\_FREQ1}
	\item \texttt{CC1100\_CONFIG\_FREQ0}
	\item \texttt{CC1100\_CONFIG\_MCSM1}
	\item \texttt{CC1100\_CONFIG\_MCSM0}
\end{itemize}
\end{center}

%%%%%%%%%%%%%%%%%%%%%%%%%%%%%%%%%%%%%%%%%%%%%%%%%%%%%%%%%%%%%%%%%%%%%%%%%%%%%%
\section{CCxx00 Radio Stack History Manifesto}
\label{sec:manifesto}

Below are some personal comments on my observations leading up to and through the development
of the CCxx00 radio stack. 

\subsection{Bluetooth}
Rincon originally pursued Bluetooth technology for basic wireless networking.  This was
abandoned as wireless sensor networks came to light and we realized Bluetooth was too power
hungry and too short range for our intended applications.

\subsection{CC1000 Nightmares}
Working with TinyOS, the team from Rincon began down the path like everyone else in the TinyOS
community: working with off-the-shelf, rather flakey hardware featuring the 
CC1000 radio at 915 MHz. Flaws in the hardware design of these off-the-shelf platforms caused mismatches in
the matching network of the radio, and this was coupled into poor antennas. Communication was horrible.
Frustrations ran high.

\subsection{Tmote Sky}
The CC1000 platform was soon abandoned as the tmote platform was released, featuring the CC2420 
2.4 GHz 802.15.4 compliant radio. Rincon adopted this new off-the-shelf tmote platform for 
deliverable systems, and made significant architecture changes to the radio stack going into 
TinyOS 2.x while doing so. We decided early on to contribute infrastructure back to the
open source community, since our business was to sell systems and solutions, not infrastructure. Life was good.

\subsection{2.4 GHz is the devil}
Basic testing revealed 2.4 GHz is an extremely poor choice for communication. We wanted hardware
to work on the ground, but we could only get a parking space distance or two on a good day. 

This led me to realize there are exactly two technical reasons why 2.4 GHz would make sense 
for any system: antenna size, and multi-country ISM band usage without hardware modification. 
Two non-technical reasons include the availability of off-the-shelf hardware, and marketing. 
The Intel culture has brainwashed everyone to think 
that more gigahertz is better. In the case of RF propagation, it's just the opposite: lower frequencies
are better. The tribulations people experienced with the CC1000 radio seemed to encourage
the wireless sensor network community to shy away from sub-GHz comms.

Your operating frequency is mutually exclusive of your data rate, which is another 
common misconception I hear quite often. To prove my point, the CC1101 radio has twice the 
data rate as the 2.4 GHz 802.15.4 radios, and the CC1101 operates down to 300 MHz. Case closed.

The downsides to 2.4 GHz communication should be obvious. WiFi interference is everywhere. Portable
phones moved away from 900 MHz to 2.4 GHz, and continue to move to other bands. 2.4 GHz is where
your microwave ovens sit. 2.4 GHz is absorbed by the ground and walls.  It does not propagate well, 
and therefore is not a good choice for wireless sensor networks in general.

If you can think of any more reasons why 2.4 GHz would make sense for any given system, please email me. 

\subsection{Dual-radio platforms}
We started getting requirements in to have a dual-radio platform.  Dual-radio platforms couldn't be purchased off-the-shelf,
so we had to spin our own. One radio continued to operate at 2.4 GHz because of the small antenna size, 
while the other radio operated the network at a much lower frequency.  The CC1100 / CC2500 radios 
were selected for this platform because they are pin and chip compatible and only require one radio driver to operate.

We ported TinyOS to SoftBaugh \cite{softbaugh} development boards and built a basic radio stack to facilitate testing
of our custom hardware after assembly.  The directory for this basic, non-fully functional stack was labeled 'ccxx00'.
The hardware platforms being developed were dubbed 'blaze', and we needed a place to store a fully functional
dual-radio platform stack. It seemed logical to call the dual-radio stack 'blaze' to match up with the hardware
it was going on.  The Blaze radio stack was born, and files labeled throughout the stack maintain this heritage.

\subsection{Iterations}
We did several revisions of the CCxx00 stack.  We blazed our own path (pun intended) of designing how a dual-radio platform
should look in TinyOS. TUnit was created just before the development of the CCxx00 stack, so unit tests were employed
through development, starting with basic SPI bus communication and verification of the radio registers. 
The first revision got basic communication up and running, but no low power communications implementations. Some areas of the radio stack had issues and needed to be redone after the first phase.  

The second iteration fixed the architectural issues and started adding in wake-on radio. As described in the
wake-on radio section, we got it working but found it didn't work too well. The chances of us getting it to work
100\% were slim given the amount of time we had for development, so we abandoned it and built BMAC. 

BMAC worked great on these platforms, so we also tried a few other low power communications layers. The good
ones still exist.

The third iteration tried to knock out bugs dealing with the radio hardware. Using SPI bus analyzers, we
were able to see our platform telling the radio to enter TX mode or RX mode, but the radio would refuse and go 
to IDLE.  This caused us to rearchitect even more of the stack to be able to reboot the radio while in the
middle of transmission or reception logic. Quite annoying.

We tried to get a layer built into the low levels of the stack to throttle back communications to the TX/RX FIFO's.
This would have allowed the radio stack to communicate packets greater than the length of the 64 byte FIFO's, but
never got it working 100\% because we didn't have the hours to finish it.

At some point, we figured out how to get the radio to properly transmit at low data rates, which we were not 
able to do before because our previous versions didn't burst-init some TEST registers. This 
spawned a fourth revision of the radio stack tailored to these low data rates. The general rule is:
every 50\% decrease in data rate results in a 50\% increase in range. Revisiting 802.15.4 which claims to be low
data rate, we see that 500, 250, or even 40 kbps is definitely not a low data rate. 1.2 kbps is. 5 kbps maybe.
Who ever used dial up internet access in the '90s with a 250 or 500 kbps data rate?  Not me.

Our networks now operate at the lowest frequency we can get our hands on, and at a low data rate. We see
700-800 feet on the ground, and up to 1 mile with nodes in the air at 10 kBaud Manchester GFSK. When we lowered the
data rate to 1.2 kBaud Manchester GFSK, it was more like 4 kilometers with one node on the roof
and the other in my hand.  No external PA's or LNA's. We did have a large
ground plane, though, made out of 4 wires that formed an X with the
monopole wire antenna sticking up through the middle. That large
RF ground plane helped our range a lot.

\subsection{Polish}
The goal of the fifth iteration of the radio stack development was to decrease the code size as much as
possible. In doing this, we used module\_memory\_usage \cite{modulememoryusage}, a script developed by Cory Sharp, to see which
modules were sucking up the most program ROM.  Turns out every time you use a 32-bit alarm, it creates
an instance of TransformAlarmC which eats away memory, and the radio stack had several of these. 
Swapping it out with a 16-bit alarm didn't change functionality, but did significantly decrease memory
footprint. 

We also extracted the parameterized interfaces throughout the stack, originally designed for the dual-radio
platform. Now, multiple radios can be selected by letting the stack dynamically switch the wiring in
the Central Wiring component.

Other modifications were made, unnecessarily functionality was separated, and the stack became a radio
stack that could be compiled for single- or multi-radio platforms.

\subsection{CCxx00 Issues}
The CC1100 radio has some issues, but once we worked past them, the radio outperformed other radios 
in the wireless sensor networking community.  One issue is the problems on the digital logic side, where
the state sometimes refuses to change. Another issue is with the wake-on radio implementation and how
it sometimes just stops working.  The radio sometimes registers a busy channel while doing a clear
channel assessment, even though the channel is definitely shown to be clear on the spectrum analyzer.
The errata is thick \cite{cc1101errata}.  For these reasons and others, the radio stack took a long time to develop.  
Everything turned out solid.

\section{Acknowledgments}
Special thanks to everyone who helped out with the development of this radio stack, whether it was 
contributing code or sending back bug reports. This includes Rincon Research Corporation for funding 
the development, Rob Muir, Jared Hill, Mark Hays, Mark Siner, Jon Wyant, Lexis Mazerski, Chris Kibbey, 
Peter Bigot, Roland Hendel, Adam Brzostowksi, Mike Claassen, WenZhan Song, Renjie Huang, Gang Lu, Miklos Maroti, 
Gustavo Litovsky, and Shaun Lawrence.

\bibliographystyle{plain}
\bibliography{ccxx00}
		
\end{document}  %End of document.
