\section{Introduction}
\label{introduction}

% keywords: Wireless Sensor Networks
The development of Wireless Sensor Networks (WSNs) \cite{1389832} has become one of the major enabling technologies for ubiquitous environments, that is, the capability of making use of seamless integrated technology in our surroundings in order to make them intelligent, thus, better serving our needs. By intelligent environment, we mean, an environment able to gather data, process local generated data, and actuate over that same environment. WSNs enables us to easily and cost-effective make this change in the environment. WSNs are composed by sensor nodes which are characterized by being low-cost, low-power, and multi-functional sensing devices, capable of performing tasks such as sensing, data processing using a micro-controller and communication through a transceiver.\\

% keywords: WSNs Applications
Typical WSNs applications are dedicated to closely observe real-world phenomena. One important operation on collecting data from a WSN is \textit{data aggregation}, whereby data reported by each sensor node is agglomerated to form a single meaningful result. The aggregation of individual sensor readings is possible only by exchanging messages that are timestamped by each sensor's local clock. This clearly mandates for a \textit{common notion of time} among the sensor nodes. However, not only WSN applications but also many of the networking protocols used in WSNs need this common notion of time. Prime examples are Medium Access Control (MAC) protocols based on Time Division Multiple Access (TDMA) or MAC protocols with coordinated wake up, like the one used in the IEEE 802.15.4 WPAN standard. Sensor nodes running a TDMA protocol need to agree on boundaries of time slots.\\

% keywords: Time Synchronization Requirements
We have seen by now that WSNs applications and network protocols have in fact requirements for a common notion of time. Protocols that provide such a common notion of time are clock synchronization protocols. In the past, researchers have developed successful  clock synchronization protocols for wired networks. These are unsuitable for a wireless sensor environment because the challenges posed by WSNs are different and manifold. Regarding the available time synchronization protocols specifically for WSN, the choice of picking the most suitable can be quite peculiar. All of them, tend to fill the necessary requirements of a WSN time synchronization protocol, but fail, by seeking time precisions that are not required by their applications. For example, Flooding Time Synchronization Protocol (FTSP) \cite{Maroti04:FTSP} manages to achieve an outstanding average precision error per hop of 0,5 $\mu$s. This achievable precision error can be seen as acceptable for most wired networks. But for WSNs, where most sensor nodes are resource constrained, typically, energy limited, the effort spending on trying to achieve such precision error is resource demanding. One can even state that most WSN applications don't need such fine-grained time precisions, but then we would be putting aside those exotic applications that in fact do need it. Nonetheless, an application should be able to be provided with the desired time precision it requires.
This situation stresses the need for the development of an adaptive approach to time synchronization in WSN, one that adapts to the time precision requirements of its application/MAC layer protocol by providing flexibility on achieving a desired time precision and at the same time adapts his resource consumption while trying to achieve this.\\

% keywords: Adaptive Time Synchronization, Motivation, Goals
Having realized the inadequacy of existing synchronization approaches, there was a need to develop Tagus Time Synchronization Protocol (TTSP), a simple, yet scalable and energy-efficient solution to the problem of timing synchronization in sensor networks that is flexible enough to meet the desired levels of time precision and algorithmic overhead. That is, applications and MAC layer protocols shall be able to make use of TTSP in a cross-layer approach, by declaring their time precision requirements directly to it. By knowing the time precision requirements that it must fulfill, TTSP will be able to adapt itself in order to assure that those requirements are fulfilled without wasting more resources than it is required.
Thus, TTSP is purposed here as an adaptive approach to time synchronization which seeks to achieve a network wide synchronization in a scalable fashion way with application/MAC layer protocol specific precision while making an efficient use of the available network resources. This means that the synchronization protocol is aware of the application/MAC layer protocol time precision requirements, preventing itself from wasting valuable network resources in order to deliver a precision that clearly exceeds the application/MAC layer protocol demand or its not even suitable. This approach shall free valuable resources and ultimately contribute to saving energy, therefore extending the network's useful life-time.\\

% keywords: Document Structure
The remainder of this paper is organized into four main sections. The following section, section two provides an overview of the related time synchronization protocols for WSNs. Section three, in turn, provides a detailed view of the proposed architecture. Section four provides an objective validation of TTSP's functionality, while, section five draws some final conclusions.\\