\begin{resumo}
%-----------------------------------------------------------
% keywords: Wireless Sensor Networks, Time Synchronization
%-----------------------------------------------------------
A necessidade de sincronização de relógios em redes de sensores sem fios é de elevada importância para a correcta identificação de eventos e a coordenação dos ciclos de actividade dos nós que constituem estas redes.
%-----------------------------------------------------------
% keywords: Precision Requirements
%-----------------------------------------------------------
Os típicos protocolos de sincronização para redes de sensores sem fios são desenhados tendo em conta um único objectivo, o de alcançar o mínimo erro de precisão entre os relógios dos vários nós. Na maior parte das aplicações e protocolos de acesso ao meio que operam neste tipo de redes, o objectivo de alcançar um erro de precisão mínimo não é de todo necessário, e o seu cumprimento pode levar a uma utilização ineficiente dos recursos que estão disponíveis pelos nós. 
%-----------------------------------------------------------
% keywords: Adaptive Time Synchronization, Cross-layer
%-----------------------------------------------------------
O Tagus Time Synchronization Protocol (TTSP), é proposto como uma alternativa aos protocolos de sincronização que estão disponíveis de momento, dado que faz uso de uma abordagem adaptativa e uma arquitectura entre-camadas com o intuito de minimizar a quantidade de recursos utilizados ao garantir uma sincronização de relógios entre nós com um erro de precisão inferior ao requisitado pela aplicação ou pelo protocolo de acesso ao meio.
%-----------------------------------------------------------
% keywords: Tagus-SensorNet
%-----------------------------------------------------------
O protocolo proposto foi implementado no popular sistema operativo para redes sem fios de sensores, TinyOS, e executado na plataforma micaZ da Crossbow. Tendo sido validado numa rede experimental de sensores presente no edíficio principal do Taguspark, campus do Instituto Superior Técnico da Universidade Técnica de Lisboa.
\end{resumo}